\paragraph{}
	Afin de communiquer avec l'application web, le client sera doté d'un module réseau qui enverra et 		
	recevra des requêtes http. Les partitions seront envoyées et reçues en format JSON afin de faciliter 
	ces échanges. \\
	Le JSON aura format semblable à ceci :
	
		\begin{tabbing}
		\kill XX\=XX\=XX\=XX\=XX\=XX\=XX\=XX\=XX\=XX\=
		\kill
			\{ \\
				\>“Name” : “string”, \\
				\>“Tempo” : 0, \\
				\>“private” : true/false \\
				\>“tunes” : [\{ \\
					\>\>“volume” : 0-1, \\
					\>\>“nb\_notes” : 0, \\
					\>\>“notes” : [\{ \\
						\>\>\>“name” : “do-la-…”, \\
						\>\>\>“frequence” : 0 \\
					\>\>\}] \\
				\>\}]	 \\
			\}
		\end{tabbing}