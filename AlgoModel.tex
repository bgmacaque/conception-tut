\begin{tabbing}

\ul{fonction} analyser(frequences : \ul{tableau entier}[0..n], n : \ul{entier}, seuilMin : \ul{entier}) : \ul{tableau Accord}\\
\ul{debut}\\
frequences <- trier(frequences)\\
i <- 0\\
complet <- faux\\
\ul{Tant que} i < n \ul{et} \ul{non} complet \ul{faire}\\
    freq <- frequences[i]\\
    \ul{Si} freq > seuilMin \ul{et} \ul{non} present(retour, freq)\\
    \ul{alors}\\
        ajouter(retour, freq)\\
        \ul{Si} taille(retour) > 5\\
        \ul{alors}\\
            complet <- vrai\\
        \ul{fsi}\\
    \ul{fsi}\\
\ul{ftantque}\\
\ul{retourne} retour\\
\ul{fin}\\
\end{tabbing}

% Algo de récupérer une note avec sa fréquence
\begin{tabbing}
\ul{fonction} recupererNote(frequence : \ul{entier}, notes : \ul{Note[0..n]}, n : \ul{entier}) : Note\\
\ul{debut}\\
min <- 0\\
max <- n\\
trouve <- faux\\
\ul{Tant que} min <= max \ul{et} \ul{non} trouve \ul{faire}\\
    indice <- (max + min) /2 \\
    frequenceNote <- notes[indice].frequence\\
    \ul{Si} frequence = frequenceNote\\
    \ul{alors}\\
        retour <- notes[indice]\\
        trouve <- vrai\\
    \ul{sinon}\\
        \ul{Si} frequence < frequenceNote\\
        \ul{alors}\\
            max <- indice\\
        \ul{sinon}\\
            min <- indice\\
        \ul{fsi}\\
    \ul{fsi}\\
\ul{ftantque}\\
\ul{Si} \ul{non} trouve\\
\ul{alors}\\
    retour <- notes[min]\\
\ul{fsi}\\
\ul{retourne} retour\\
\ul{fin}\\
\end{tabbing}


\begin{tabbing}
\ul{fonction} trier(frequences : \ul{tableau eniter}[0..n], n\ul{entier})
\end{tabbing}


\begin{tabbing}
\ul{fonction} montrerNote(mic : \ul{entreeMicro})\\
\ul{debut}\\
accorder <- vrai\\
\ul{Tant que}\\
freqs <- recupererFrequences(mic)\\
freqs <- trier(freqs)\\
\ul{ecrire} recupererNote(freqs[0])\\
\ul{ftantque}\\
\ul{fin}\\
\end{tabbing}

%Enregistrement d'une partition
\begin{tabbing}
\ul{fonction} enregistrer(mic : \ul{entreeMicro}, seuilHaut :  \ul{entier}, seuilBas : \ul{entier}, tempo : \ul{entier}) : \ul{Partition}\\
\ul{debut}\\
enregistrer <- vrai\\
j <- 0\\
\ul{Tant que} enregistrer \ul{faire}\\
    frequences <- recupererFrequences(mic)\\
    frequences <- trier(frequences)\\
    complet <- faux\\
    fini <- faux\\
    i <- 0 \\
    \ul{Tant que} \ul{non} complet \ul{et} \ul{non} fini \ul{faire}\\
        frequence <- frequences[i]
        \ul{si} frequence > seuilHaut\\
        \ul{alors}\\
            ajouter(freqs, accord)\\
            j <- j + 1\\
        \ul{sinon}\\
            \ul{si} frequence > seuilBas\\
            \ul{alors}\\ 
                ajouterTemps(retour[j], tempo / 16)\\
            \ul{sinon}\\
                fini <- vrai\\
        \ul{fsi}
    \ul{ftantque}\\     
    i <- i + 1\\  
    attendre(tempo / 16)\\
\ul{ftantque}\\
\ul{retourne} retour \\
\ul{fin}\\
\end{tabbing}

Nous attendons pendant tempo / 16 car c'est la plus petite unité de temps représentable dans une partition. L'utilisateur doit rentrer le tempo dans lequel il compte jouer.