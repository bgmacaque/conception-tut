\paragraph{}
La lecture d'une partition faisait partie d'une des fonctionnalités à priorité haute de l'application. Après 
de nombreuses recherches nous n'avons pas pu récupérer des fichiers MIDI correspondant aux sons de chaques 
notes de musique possible pour permettre la lecture de l'application, et c'est pour cela que la fonctionnalité 
n'a pas été réalisée.

\paragraph{}
Nous souhaitions faire apparaitre les notes dans la partition en temps réel, mais cela a été impossible à cause 
du fait que l'enregistrement de la partition se fait dans un Thread, et la création de nouveaux objets est 
interdite par Qt. Nous avions donc une erreur, et une améliorations possible serait de permettre cette fonctionnalité.

\paragraph{}
Nous nous sommes surtout concentré sur l'enregistrement de partition à partir du micro, nous n'avons donc pas fini 
de développer complètement la modification des partitions après cet enregistrement.

\paragraph{}


\paragraph{}
La répartition des taches pour le client a un peu été changée à la fin du projet. Nous avions pensé que le dévelopemment 
du modèle serait aussi important que celui de la partie graphique, or en fin de projet, nous avons été plus nombreux à 
développer la partie graphique. Cela est surtout du au fait qu'il fallait lier la partie graphique et la partie des données, 
et nous avons donc plus pu travailler en collaboration à ce moment là. L'intégration n'a pas engendré trop de difficultés, car 
nous avions conçu dès le début les deux parties de sorte que tous les éléments soient là pour nous faciliter la jointure.