\paragraph{}
La technologie NodeJS est une technologie très intéressante car elle possède une forte communauté et les frameworks qui en découlent sont très puissants. \\
Nous avons utilisé Grunt qui est un gestionnaire de tache, ce programme permettait de lancer le serveur à l'aide d'une simple ligne de commande qui se résumait à un mot : grunt. De plus, il permettait de notifier le client à chaque modification du serveur, ainsi à chaque modification du serveur, le client se raffraichit automatiquement. \\
Nous avons utilisé le framework Compass pour le CSS, ce framework possède des mixins qui permettent d'améliorer la vitesse de production. De plus, cette surcouche au CSS ajoute la possibilité d'utiliser les variables et de structurer le code plus simplement. \\ De plus, ce framework a été combiné à celui de Bootstrap pour permettre la gestion du responsive design. \\
Du côté du modèle, Sequelize est un ORM jeune mais c'est le seul ORM convenable sur NodeJS, il permet donc d'utiliser le pattern ActiveRecord sans trop de problème. (Il crée parfois des colonnes inatendues dans les tables MySQL). \\
Pour la base de données en elle-même, nous avons utilisé MySQL, nous avons vu cette technologie à l'IUT et c'est pourquoi nous n'avons pas rencontré de problème avec celle-ci. \\
Pour générer la vue (le code HTML), nous avons utilisé HandleBarsJS qui est un moteur de templates très puissant et customisable. Ainsi nous avons ajouté des fonctions à ce dernier pour qu'il convienne mieux au projet. \\
Enfin, nous avons utilisé les websockets pour permettre la mise a jour en instantannée de certaines informations. Cette technologie a été difficile à mettre en place à cause de la gestion de variables de sessions. Il fallait donc envoyer en plus des informations de base, le cookie correspondant à la session.
