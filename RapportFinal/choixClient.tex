\paragraph{}
Le langage C++ a été un choix apprécié pour l'élaboration du client. Le langage est puissant et rapide ce qui permet un 
traitement des fréquences très rapide, malgré une quantité d'informations importante. L'étude d'un nouveau langage au sein 
de ce projet a donc été une chose bénéfique car le langage a directement été appliqué à la pratique. 

\paragraph{}
De plus, l'utilisation du framework Qt s'est avéré être un choix judicieux nous permettant non seulement de développer l'interface graphique du client, mais nous fournissant aussi des outils pour manipuler la base de données du site web à distance à partir de l'application. C'est grâce à cela que l'utilisateur peut donc mettre une partition à jour depuis le client. La connexion de l'application à la base de données se fait par le biais du driver adapté 
à la base de données utilisée. Il serait donc également aisé de changer de système de gestion de base de données si on le souhaiterait. 

\paragraph{}
Pour l'analyse des sons entrant, le choix de la bibliothèque FMOD nous a été très utile durant la période de développement. Une fois installée, 
elle permet la récupération de toutes les fréquences entrant par le micro, et cela simplement. \\
Cependant elle n'est pas exempte de défaut et nous n'avons pas réussi à l'utiliser convenablement sur tous les systèmes d'exploitation, ce qui empêche de rendre notre application multiplate-forme comme on l'espérait initialement. L'installation de cette librairie est difficile et 
très peu de guide nous informe sur les problèmes rencontrés, surtout qu'il faut de plus configurer son utilisation à Qt, différent de son utilisation 
en ligne de commande. 

