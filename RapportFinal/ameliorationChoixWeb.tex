\paragraph{}
Nous voulions utiliser le framework Ember ou Angular pour coder l'application web. Cependant, nous avons fait le choix d'utiliser jQuery et un moteur de template pour éviter d'apprendre encore une nouvelle technologie que nous ne métrisions pas. \\
Ainsi, les échanges entre le serveur et le client pourraient être améliorés grâce à l'utilisation d'un framework plus complexe. \\
Étant donné le temps passé à l'apprentissage de NodeJS, nous n'avons pas eu le temps de coder toutes les fonctionnalitées voulues au départ. \\
Par exemple, l'envoie de messages entre utilisateur et les notifications en instantannée ne sont pas implantées. Par ailleurs, les notifications en temps réel (à l'aide des websockets) nécessiteraient un serveur suffisamment puissant pour qu'elles puissent fonctionner sans problème. \\
Puis, l'affichage et l'écoute des partitions pourraient être améliorés.
Nous voulons utiliser un framework déjà existant pour afficher les partitions et ensuite coder notre propre framework qui s'occupera de parser les informations des partitions disponibles sous le format JSON. \\
\paragraph{}
Au niveau de la sécurité, certains points sont à améliorer. Il faut augmenter les vérifications lors des envoies de partitions ou d'images de profil. De plus on pourrait aussi rajouter un jeton particulier à chaque formulaire de l'application, ce jeton permettrait de vérifier si le formulaire utilisé n'est pas un formulaire détourné.
\paragraph{}
La possibilité d'ajouter en favoris une partition est aussi une amélioration possible (mais non essentielle).

