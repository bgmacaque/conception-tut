\paragraph{}
La structure du site suivra le modèle MVC.
Des tests unitaires seront utilisés pour s'assurer du bon fonctionnement de l'application.
\\
\\
Le site sera donc divisé en trois principales parties : \\
-Views \\
-Controllers \\ 
-Models \\
Plus une partie qui contiendra les routes (ie les chemins d'accès aux différents controllers)
\\

Views : \\ 
-home => représentera l'accueuil du site
-compte => Profil utilisateur \\
-groupe => Profil du groupe \\
-listing => Listing des différents partitions avec option de recherche\\
-topbar => barre du haut qui sera incluse sur toutes les pages\\
-tabpage => page principale d'une partition\\

Controllers: \\
Les controllers seront utilisés pour faire le lien entre le modèle et les différentes vues. De plus ils permettront de gérer les fonctionnalités spécifiques au site. \\
home.js => controller général \\
profil.js => controller du compte qui gère le profil utilisateur\\
groupe.js => controller qui gère les groupes utilisateurs \\
listing.js => controller qui gère la liste des partitions ainsi que la fonction recherche dans celles-ci\\
topbare.js => permet de gérer la barre de menu présente sur le haut du site\\
partition.js  => permet de gérer les fonctionnalités d'une partition en particulier\\
partition-page.js => controller qui gère la page d'une partition\\

Les modèles suivent l'architecture Active Record. \\
Ils seront donc liés à la base de données (cf MCD et MLD). \\
Une table équivaut donc à un module JavaScript. \\
Ces modules contiendront toutes les méthodes nécessaires au traitement et à la modification du modèle (findById,insert,delete,update ...) et suivront donc l'approche vue en cours. \\
\\

Routes: \\
Chaque lien est une route qui peut contenir une action et qui actionne une méthode du controller lié à la route. \\
Une route est donc un chemin représenté par un URL. \\
Ces routes sont donc le coeur du fonctionnement de l'application Web\\

