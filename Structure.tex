\paragraph{}
La structure du site suivra le modèle MVC.
Des tests unitaires seront utilisés pour s'assurer du bon fonctionnement de l'application.
\\
\\
Le site sera donc divisé en trois principales parties : \\
-Views \\
-Controllers \\ 
-Models \\
Plus une partie qui contiendra les routes (ie les chemins d'accès aux différents controllers)
\\

Views : \\ 
-home => représentera l'accueuil du site
-compte => Profil utilisateur \\
-groupe => Profil du groupe \\
-listing => Listing des différents partitions avec option de recherche\\
-topbar => barre du haut qui sera incluse sur toutes les pages\\
-tabpage => page principale d'une partition\\

Controllers: \\
Les controllers seront utilisés pour faire le lien entre le modèle et les différentes vues. De plus ils permettront de gérer les fonctionnalités spécifiques au site. \\
home.js || controlleur général \\
compte.js || controlleur du compte => gère les effets \\
groupe.js || controlleur gère effet \\
listing.js || gèrent effet \\
bare.js \\
partition.js \\
partition-page.js \\


home.js => gestion des animations + gestion des datas à afficher \\



DATA => voir base de données \\
Active record => une table = un module \\
-methode de base (findAll,byId,insert,update,delete)+ sorted by ...  \\

les routes express gèrent les chemins de navigations  \\


Routes:
chaque lien est une route qui peut contenir une action et qui actionne une méthode du controller lié à la route \\
